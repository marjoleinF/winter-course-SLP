% Options for packages loaded elsewhere
\PassOptionsToPackage{unicode}{hyperref}
\PassOptionsToPackage{hyphens}{url}
%
\documentclass[
]{article}
\usepackage{amsmath,amssymb}
\usepackage{iftex}
\ifPDFTeX
  \usepackage[T1]{fontenc}
  \usepackage[utf8]{inputenc}
  \usepackage{textcomp} % provide euro and other symbols
\else % if luatex or xetex
  \usepackage{unicode-math} % this also loads fontspec
  \defaultfontfeatures{Scale=MatchLowercase}
  \defaultfontfeatures[\rmfamily]{Ligatures=TeX,Scale=1}
\fi
\usepackage{lmodern}
\ifPDFTeX\else
  % xetex/luatex font selection
\fi
% Use upquote if available, for straight quotes in verbatim environments
\IfFileExists{upquote.sty}{\usepackage{upquote}}{}
\IfFileExists{microtype.sty}{% use microtype if available
  \usepackage[]{microtype}
  \UseMicrotypeSet[protrusion]{basicmath} % disable protrusion for tt fonts
}{}
\makeatletter
\@ifundefined{KOMAClassName}{% if non-KOMA class
  \IfFileExists{parskip.sty}{%
    \usepackage{parskip}
  }{% else
    \setlength{\parindent}{0pt}
    \setlength{\parskip}{6pt plus 2pt minus 1pt}}
}{% if KOMA class
  \KOMAoptions{parskip=half}}
\makeatother
\usepackage{xcolor}
\usepackage[margin=1in]{geometry}
\usepackage{color}
\usepackage{fancyvrb}
\newcommand{\VerbBar}{|}
\newcommand{\VERB}{\Verb[commandchars=\\\{\}]}
\DefineVerbatimEnvironment{Highlighting}{Verbatim}{commandchars=\\\{\}}
% Add ',fontsize=\small' for more characters per line
\usepackage{framed}
\definecolor{shadecolor}{RGB}{248,248,248}
\newenvironment{Shaded}{\begin{snugshade}}{\end{snugshade}}
\newcommand{\AlertTok}[1]{\textcolor[rgb]{0.94,0.16,0.16}{#1}}
\newcommand{\AnnotationTok}[1]{\textcolor[rgb]{0.56,0.35,0.01}{\textbf{\textit{#1}}}}
\newcommand{\AttributeTok}[1]{\textcolor[rgb]{0.13,0.29,0.53}{#1}}
\newcommand{\BaseNTok}[1]{\textcolor[rgb]{0.00,0.00,0.81}{#1}}
\newcommand{\BuiltInTok}[1]{#1}
\newcommand{\CharTok}[1]{\textcolor[rgb]{0.31,0.60,0.02}{#1}}
\newcommand{\CommentTok}[1]{\textcolor[rgb]{0.56,0.35,0.01}{\textit{#1}}}
\newcommand{\CommentVarTok}[1]{\textcolor[rgb]{0.56,0.35,0.01}{\textbf{\textit{#1}}}}
\newcommand{\ConstantTok}[1]{\textcolor[rgb]{0.56,0.35,0.01}{#1}}
\newcommand{\ControlFlowTok}[1]{\textcolor[rgb]{0.13,0.29,0.53}{\textbf{#1}}}
\newcommand{\DataTypeTok}[1]{\textcolor[rgb]{0.13,0.29,0.53}{#1}}
\newcommand{\DecValTok}[1]{\textcolor[rgb]{0.00,0.00,0.81}{#1}}
\newcommand{\DocumentationTok}[1]{\textcolor[rgb]{0.56,0.35,0.01}{\textbf{\textit{#1}}}}
\newcommand{\ErrorTok}[1]{\textcolor[rgb]{0.64,0.00,0.00}{\textbf{#1}}}
\newcommand{\ExtensionTok}[1]{#1}
\newcommand{\FloatTok}[1]{\textcolor[rgb]{0.00,0.00,0.81}{#1}}
\newcommand{\FunctionTok}[1]{\textcolor[rgb]{0.13,0.29,0.53}{\textbf{#1}}}
\newcommand{\ImportTok}[1]{#1}
\newcommand{\InformationTok}[1]{\textcolor[rgb]{0.56,0.35,0.01}{\textbf{\textit{#1}}}}
\newcommand{\KeywordTok}[1]{\textcolor[rgb]{0.13,0.29,0.53}{\textbf{#1}}}
\newcommand{\NormalTok}[1]{#1}
\newcommand{\OperatorTok}[1]{\textcolor[rgb]{0.81,0.36,0.00}{\textbf{#1}}}
\newcommand{\OtherTok}[1]{\textcolor[rgb]{0.56,0.35,0.01}{#1}}
\newcommand{\PreprocessorTok}[1]{\textcolor[rgb]{0.56,0.35,0.01}{\textit{#1}}}
\newcommand{\RegionMarkerTok}[1]{#1}
\newcommand{\SpecialCharTok}[1]{\textcolor[rgb]{0.81,0.36,0.00}{\textbf{#1}}}
\newcommand{\SpecialStringTok}[1]{\textcolor[rgb]{0.31,0.60,0.02}{#1}}
\newcommand{\StringTok}[1]{\textcolor[rgb]{0.31,0.60,0.02}{#1}}
\newcommand{\VariableTok}[1]{\textcolor[rgb]{0.00,0.00,0.00}{#1}}
\newcommand{\VerbatimStringTok}[1]{\textcolor[rgb]{0.31,0.60,0.02}{#1}}
\newcommand{\WarningTok}[1]{\textcolor[rgb]{0.56,0.35,0.01}{\textbf{\textit{#1}}}}
\usepackage{graphicx}
\makeatletter
\def\maxwidth{\ifdim\Gin@nat@width>\linewidth\linewidth\else\Gin@nat@width\fi}
\def\maxheight{\ifdim\Gin@nat@height>\textheight\textheight\else\Gin@nat@height\fi}
\makeatother
% Scale images if necessary, so that they will not overflow the page
% margins by default, and it is still possible to overwrite the defaults
% using explicit options in \includegraphics[width, height, ...]{}
\setkeys{Gin}{width=\maxwidth,height=\maxheight,keepaspectratio}
% Set default figure placement to htbp
\makeatletter
\def\fps@figure{htbp}
\makeatother
\setlength{\emergencystretch}{3em} % prevent overfull lines
\providecommand{\tightlist}{%
  \setlength{\itemsep}{0pt}\setlength{\parskip}{0pt}}
\setcounter{secnumdepth}{-\maxdimen} % remove section numbering
\ifLuaTeX
  \usepackage{selnolig}  % disable illegal ligatures
\fi
\IfFileExists{bookmark.sty}{\usepackage{bookmark}}{\usepackage{hyperref}}
\IfFileExists{xurl.sty}{\usepackage{xurl}}{} % add URL line breaks if available
\urlstyle{same}
\hypersetup{
  pdftitle={Answers to exercises Session 5},
  pdfauthor={Marjolein Fokkema},
  hidelinks,
  pdfcreator={LaTeX via pandoc}}

\title{Answers to exercises Session 5}
\author{Marjolein Fokkema}
\date{}

\begin{document}
\maketitle

\hypertarget{exercise-1}{%
\section{Exercise 1}\label{exercise-1}}

Read in data:

\begin{Shaded}
\begin{Highlighting}[]
\FunctionTok{load}\NormalTok{(}\StringTok{"MASQ.Rda"}\NormalTok{)}
\FunctionTok{set.seed}\NormalTok{(}\DecValTok{1}\NormalTok{)}
\NormalTok{train }\OtherTok{\textless{}{-}} \FunctionTok{sample}\NormalTok{(}\DecValTok{1}\SpecialCharTok{:}\FunctionTok{nrow}\NormalTok{(MASQ), }\AttributeTok{size =} \FunctionTok{nrow}\NormalTok{(MASQ)}\SpecialCharTok{*}\NormalTok{.}\DecValTok{8}\NormalTok{)}
\FunctionTok{summary}\NormalTok{(MASQ)}
\end{Highlighting}
\end{Shaded}

\begin{verbatim}
##  D_DEPDYS       AD               AA             GDD             GDA      
##  0:1927   Min.   : 26.00   Min.   :17.00   Min.   :12.00   Min.   :11.0  
##  1:1670   1st Qu.: 64.00   1st Qu.:22.00   1st Qu.:20.00   1st Qu.:19.0  
##           Median : 77.00   Median :28.00   Median :29.00   Median :24.0  
##           Mean   : 75.05   Mean   :32.01   Mean   :30.64   Mean   :25.4  
##           3rd Qu.: 88.00   3rd Qu.:39.00   3rd Qu.:40.00   3rd Qu.:31.0  
##           Max.   :110.00   Max.   :83.00   Max.   :60.00   Max.   :54.0  
##       GDM          leeftijd    geslacht
##  Min.   :15.0   Min.   :17.0   m:1317  
##  1st Qu.:31.0   1st Qu.:28.0   v:2280  
##  Median :40.0   Median :38.0           
##  Mean   :40.6   Mean   :38.8           
##  3rd Qu.:50.0   3rd Qu.:48.0           
##  Max.   :75.0   Max.   :91.0
\end{verbatim}

\begin{Shaded}
\begin{Highlighting}[]
\FunctionTok{round}\NormalTok{(}\FunctionTok{cor}\NormalTok{(MASQ[train, }\FunctionTok{sapply}\NormalTok{(MASQ, is.numeric)]), }\AttributeTok{digits =} \DecValTok{2}\NormalTok{)}
\end{Highlighting}
\end{Shaded}

\begin{verbatim}
##            AD   AA   GDD   GDA   GDM leeftijd
## AD       1.00 0.51  0.79  0.61  0.74     0.01
## AA       0.51 1.00  0.58  0.79  0.70     0.01
## GDD      0.79 0.58  1.00  0.72  0.81    -0.07
## GDA      0.61 0.79  0.72  1.00  0.80    -0.05
## GDM      0.74 0.70  0.81  0.80  1.00    -0.04
## leeftijd 0.01 0.01 -0.07 -0.05 -0.04     1.00
\end{verbatim}

Fit a smoothing spline of the \texttt{AD} variable to predict
\texttt{D\_DEPDYS}:

\begin{Shaded}
\begin{Highlighting}[]
\FunctionTok{library}\NormalTok{(}\StringTok{"mgcv"}\NormalTok{)}
\NormalTok{GAM }\OtherTok{\textless{}{-}} \FunctionTok{gam}\NormalTok{(D\_DEPDYS }\SpecialCharTok{\textasciitilde{}} \FunctionTok{s}\NormalTok{(AD, }\AttributeTok{bs =} \StringTok{"cr"}\NormalTok{), }
           \AttributeTok{data =}\NormalTok{ MASQ[train, ], }\AttributeTok{method =} \StringTok{"REML"}\NormalTok{, }\AttributeTok{family =} \StringTok{"binomial"}\NormalTok{)}
\FunctionTok{summary}\NormalTok{(GAM)}
\end{Highlighting}
\end{Shaded}

\begin{verbatim}
## 
## Family: binomial 
## Link function: logit 
## 
## Formula:
## D_DEPDYS ~ s(AD, bs = "cr")
## 
## Parametric coefficients:
##             Estimate Std. Error z value Pr(>|z|)    
## (Intercept) -0.24089    0.04751   -5.07 3.98e-07 ***
## ---
## Signif. codes:  0 '***' 0.001 '**' 0.01 '*' 0.05 '.' 0.1 ' ' 1
## 
## Approximate significance of smooth terms:
##        edf Ref.df Chi.sq p-value    
## s(AD) 4.09  5.041  672.4  <2e-16 ***
## ---
## Signif. codes:  0 '***' 0.001 '**' 0.01 '*' 0.05 '.' 0.1 ' ' 1
## 
## R-sq.(adj) =  0.297   Deviance explained = 23.7%
## -REML = 1522.2  Scale est. = 1         n = 2877
\end{verbatim}

\begin{Shaded}
\begin{Highlighting}[]
\FunctionTok{plot}\NormalTok{(GAM, }\AttributeTok{residuals =} \ConstantTok{TRUE}\NormalTok{)}
\end{Highlighting}
\end{Shaded}

\includegraphics{Answers-to-exercises_files/figure-latex/unnamed-chunk-2-1.pdf}

Inspect the basis functions that were created for \texttt{AD}:

\begin{Shaded}
\begin{Highlighting}[]
\NormalTok{mod\_mat }\OtherTok{\textless{}{-}} \FunctionTok{model.matrix}\NormalTok{(GAM)}
\FunctionTok{matplot}\NormalTok{(}\FunctionTok{sort}\NormalTok{(MASQ}\SpecialCharTok{$}\NormalTok{AD[train]), mod\_mat[}\FunctionTok{order}\NormalTok{(MASQ}\SpecialCharTok{$}\NormalTok{AD[train]), ], }\AttributeTok{type =} \StringTok{"l"}\NormalTok{,}
        \AttributeTok{xlab =} \StringTok{"AD"}\NormalTok{, }\AttributeTok{ylab =} \StringTok{"Basis function"}\NormalTok{)}
\end{Highlighting}
\end{Shaded}

\includegraphics{Answers-to-exercises_files/figure-latex/unnamed-chunk-3-1.pdf}

\hypertarget{exercise-2-multiple-predictors}{%
\section{Exercise 2: Multiple
predictors}\label{exercise-2-multiple-predictors}}

\begin{Shaded}
\begin{Highlighting}[]
\FunctionTok{library}\NormalTok{(}\StringTok{"mgcv"}\NormalTok{)}
\NormalTok{GAM }\OtherTok{\textless{}{-}} \FunctionTok{gam}\NormalTok{(D\_DEPDYS }\SpecialCharTok{\textasciitilde{}} \FunctionTok{s}\NormalTok{(AD) }\SpecialCharTok{+} \FunctionTok{s}\NormalTok{(AA) }\SpecialCharTok{+} \FunctionTok{s}\NormalTok{(GDD) }\SpecialCharTok{+} \FunctionTok{s}\NormalTok{(GDA) }\SpecialCharTok{+} \FunctionTok{s}\NormalTok{(GDM) }\SpecialCharTok{+} \FunctionTok{s}\NormalTok{(leeftijd) }\SpecialCharTok{+}\NormalTok{ geslacht, }
           \AttributeTok{data =}\NormalTok{ MASQ[train, ], }\AttributeTok{method =} \StringTok{"REML"}\NormalTok{, }\AttributeTok{family =} \StringTok{"binomial"}\NormalTok{)}
\FunctionTok{summary}\NormalTok{(GAM)}
\end{Highlighting}
\end{Shaded}

\begin{verbatim}
## 
## Family: binomial 
## Link function: logit 
## 
## Formula:
## D_DEPDYS ~ s(AD) + s(AA) + s(GDD) + s(GDA) + s(GDM) + s(leeftijd) + 
##     geslacht
## 
## Parametric coefficients:
##             Estimate Std. Error z value Pr(>|z|)    
## (Intercept) -0.33822    0.07755  -4.361 1.29e-05 ***
## geslachtv    0.13314    0.09517   1.399    0.162    
## ---
## Signif. codes:  0 '***' 0.001 '**' 0.01 '*' 0.05 '.' 0.1 ' ' 1
## 
## Approximate significance of smooth terms:
##               edf Ref.df  Chi.sq  p-value    
## s(AD)       4.243  5.238 148.910  < 2e-16 ***
## s(AA)       1.001  1.002   0.115  0.73591    
## s(GDD)      2.185  2.793  11.185  0.00971 ** 
## s(GDA)      3.855  4.806  33.279 4.62e-06 ***
## s(GDM)      1.001  1.001  52.723  < 2e-16 ***
## s(leeftijd) 3.050  3.817  17.530  0.00137 ** 
## ---
## Signif. codes:  0 '***' 0.001 '**' 0.01 '*' 0.05 '.' 0.1 ' ' 1
## 
## R-sq.(adj) =  0.331   Deviance explained =   27%
## -REML = 1478.1  Scale est. = 1         n = 2877
\end{verbatim}

\begin{Shaded}
\begin{Highlighting}[]
\FunctionTok{par}\NormalTok{(}\AttributeTok{mfrow =} \FunctionTok{c}\NormalTok{(}\DecValTok{2}\NormalTok{, }\DecValTok{3}\NormalTok{))}
\FunctionTok{plot}\NormalTok{(GAM)}
\end{Highlighting}
\end{Shaded}

\includegraphics{Answers-to-exercises_files/figure-latex/unnamed-chunk-4-1.pdf}

We compute the mean squared error and misclassification rate using
predicted probabilities, for both training and test observations:

\begin{Shaded}
\begin{Highlighting}[]
\NormalTok{y\_train }\OtherTok{\textless{}{-}} \FunctionTok{as.numeric}\NormalTok{(MASQ[train, }\StringTok{"D\_DEPDYS"}\NormalTok{]) }\SpecialCharTok{{-}} \DecValTok{1}
\NormalTok{y\_test }\OtherTok{\textless{}{-}} \FunctionTok{as.numeric}\NormalTok{(MASQ[}\SpecialCharTok{{-}}\NormalTok{train, }\StringTok{"D\_DEPDYS"}\NormalTok{]) }\SpecialCharTok{{-}} \DecValTok{1}

\DocumentationTok{\#\# Training data}
\NormalTok{GAM\_preds\_train }\OtherTok{\textless{}{-}} \FunctionTok{predict}\NormalTok{(GAM, }\AttributeTok{newdata =}\NormalTok{ MASQ[train, ], }\AttributeTok{type =} \StringTok{"response"}\NormalTok{)}
\FunctionTok{mean}\NormalTok{((y\_train }\SpecialCharTok{{-}}\NormalTok{ GAM\_preds\_train)}\SpecialCharTok{\^{}}\DecValTok{2}\NormalTok{) }\DocumentationTok{\#\# Brier score}
\end{Highlighting}
\end{Shaded}

\begin{verbatim}
## [1] 0.1653101
\end{verbatim}

\begin{Shaded}
\begin{Highlighting}[]
\NormalTok{tab\_train }\OtherTok{\textless{}{-}} \FunctionTok{prop.table}\NormalTok{(}\FunctionTok{table}\NormalTok{(MASQ[train, }\StringTok{"D\_DEPDYS"}\NormalTok{], GAM\_preds\_train }\SpecialCharTok{\textgreater{}}\NormalTok{ .}\DecValTok{5}\NormalTok{)) }\DocumentationTok{\#\# confusion matrix}
\NormalTok{tab\_train}
\end{Highlighting}
\end{Shaded}

\begin{verbatim}
##    
##         FALSE      TRUE
##   0 0.4174487 0.1223497
##   1 0.1160932 0.3441084
\end{verbatim}

\begin{Shaded}
\begin{Highlighting}[]
\DecValTok{1} \SpecialCharTok{{-}} \FunctionTok{sum}\NormalTok{(}\FunctionTok{diag}\NormalTok{(tab\_train)) }\DocumentationTok{\#\# MCR}
\end{Highlighting}
\end{Shaded}

\begin{verbatim}
## [1] 0.2384428
\end{verbatim}

\begin{Shaded}
\begin{Highlighting}[]
\DocumentationTok{\#\# Test data}
\NormalTok{GAM\_preds\_test }\OtherTok{\textless{}{-}} \FunctionTok{predict}\NormalTok{(GAM, }\AttributeTok{newdata =}\NormalTok{ MASQ[}\SpecialCharTok{{-}}\NormalTok{train, ], }\AttributeTok{type =} \StringTok{"response"}\NormalTok{)}
\FunctionTok{mean}\NormalTok{((y\_test }\SpecialCharTok{{-}}\NormalTok{ GAM\_preds\_test)}\SpecialCharTok{\^{}}\DecValTok{2}\NormalTok{) }\DocumentationTok{\#\# Brier score}
\end{Highlighting}
\end{Shaded}

\begin{verbatim}
## [1] 0.1666467
\end{verbatim}

\begin{Shaded}
\begin{Highlighting}[]
\NormalTok{tab\_test }\OtherTok{\textless{}{-}} \FunctionTok{prop.table}\NormalTok{(}\FunctionTok{table}\NormalTok{(MASQ[}\SpecialCharTok{{-}}\NormalTok{train, }\StringTok{"D\_DEPDYS"}\NormalTok{], GAM\_preds\_test }\SpecialCharTok{\textgreater{}}\NormalTok{ .}\DecValTok{5}\NormalTok{)) }\DocumentationTok{\#\# confusion matrix}
\NormalTok{tab\_test}
\end{Highlighting}
\end{Shaded}

\begin{verbatim}
##    
##         FALSE      TRUE
##   0 0.4027778 0.1166667
##   1 0.1236111 0.3569444
\end{verbatim}

\begin{Shaded}
\begin{Highlighting}[]
\DecValTok{1} \SpecialCharTok{{-}} \FunctionTok{sum}\NormalTok{(}\FunctionTok{diag}\NormalTok{(tab\_test)) }\DocumentationTok{\#\# MCR}
\end{Highlighting}
\end{Shaded}

\begin{verbatim}
## [1] 0.2402778
\end{verbatim}

The Brier score and confusion matrices are quite similar between
training and test data, indicating little overfitting.

\begin{Shaded}
\begin{Highlighting}[]
\DocumentationTok{\#\# Or, compute ROC curve}
\FunctionTok{library}\NormalTok{(}\StringTok{"pROC"}\NormalTok{)}
\FunctionTok{plot}\NormalTok{(}\FunctionTok{roc}\NormalTok{(}\AttributeTok{resp =}\NormalTok{ y\_test, }\AttributeTok{pred =}\NormalTok{ GAM\_preds\_test))}
\end{Highlighting}
\end{Shaded}

\includegraphics{Answers-to-exercises_files/figure-latex/unnamed-chunk-6-1.pdf}

\begin{Shaded}
\begin{Highlighting}[]
\FunctionTok{auc}\NormalTok{(}\AttributeTok{resp =}\NormalTok{ y\_test, }\AttributeTok{pred =}\NormalTok{ GAM\_preds\_test)}
\end{Highlighting}
\end{Shaded}

\begin{verbatim}
## Area under the curve: 0.8295
\end{verbatim}

\newpage

\hypertarget{exercise-4-fit-a-conditional-inference-tree}{%
\section{Exercise 4: Fit a conditional inference
tree}\label{exercise-4-fit-a-conditional-inference-tree}}

\begin{Shaded}
\begin{Highlighting}[]
\FunctionTok{library}\NormalTok{(}\StringTok{"partykit"}\NormalTok{)}
\NormalTok{ct }\OtherTok{\textless{}{-}} \FunctionTok{ctree}\NormalTok{(D\_DEPDYS }\SpecialCharTok{\textasciitilde{}}\NormalTok{ . , }\AttributeTok{data =}\NormalTok{ MASQ[train, ])}
\FunctionTok{plot}\NormalTok{(ct, }\AttributeTok{gp =} \FunctionTok{gpar}\NormalTok{(}\AttributeTok{cex =}\NormalTok{ .}\DecValTok{5}\NormalTok{))}
\end{Highlighting}
\end{Shaded}

\includegraphics{Answers-to-exercises_files/figure-latex/unnamed-chunk-7-1.pdf}

The conditional inference tree indicates a positive effect of the AD,
GDM and GDD subscales on the probability of having a depressive /
dysthymic disorder.

\begin{Shaded}
\begin{Highlighting}[]
\DocumentationTok{\#\# Training data}
\NormalTok{ct\_preds\_train }\OtherTok{\textless{}{-}} \FunctionTok{predict}\NormalTok{(ct, }\AttributeTok{newdata =}\NormalTok{ MASQ[train, ], }\AttributeTok{type =} \StringTok{"prob"}\NormalTok{)[ , }\DecValTok{2}\NormalTok{]}
\FunctionTok{mean}\NormalTok{((y\_train }\SpecialCharTok{{-}}\NormalTok{ ct\_preds\_train)}\SpecialCharTok{\^{}}\DecValTok{2}\NormalTok{) }\DocumentationTok{\#\# Brier score}
\end{Highlighting}
\end{Shaded}

\begin{verbatim}
## [1] 0.1705674
\end{verbatim}

\begin{Shaded}
\begin{Highlighting}[]
\NormalTok{tab\_train }\OtherTok{\textless{}{-}} \FunctionTok{prop.table}\NormalTok{(}\FunctionTok{table}\NormalTok{(MASQ[train, }\StringTok{"D\_DEPDYS"}\NormalTok{], ct\_preds\_train }\SpecialCharTok{\textgreater{}}\NormalTok{ .}\DecValTok{5}\NormalTok{)) }\DocumentationTok{\#\# confusion matrix}
\DecValTok{1} \SpecialCharTok{{-}} \FunctionTok{sum}\NormalTok{(}\FunctionTok{diag}\NormalTok{(tab\_train)) }\DocumentationTok{\#\# MCR}
\end{Highlighting}
\end{Shaded}

\begin{verbatim}
## [1] 0.2457421
\end{verbatim}

\begin{Shaded}
\begin{Highlighting}[]
\DocumentationTok{\#\# Test data}
\NormalTok{y\_test }\OtherTok{\textless{}{-}} \FunctionTok{as.numeric}\NormalTok{(MASQ[}\SpecialCharTok{{-}}\NormalTok{train, }\StringTok{"D\_DEPDYS"}\NormalTok{]) }\SpecialCharTok{{-}} \DecValTok{1}
\NormalTok{ct\_preds\_test }\OtherTok{\textless{}{-}} \FunctionTok{predict}\NormalTok{(ct, }\AttributeTok{newdata =}\NormalTok{ MASQ[}\SpecialCharTok{{-}}\NormalTok{train, ], }\AttributeTok{type =} \StringTok{"prob"}\NormalTok{)[ , }\DecValTok{2}\NormalTok{] }
\FunctionTok{mean}\NormalTok{((y\_test }\SpecialCharTok{{-}}\NormalTok{ ct\_preds\_test)}\SpecialCharTok{\^{}}\DecValTok{2}\NormalTok{) }\DocumentationTok{\#\# Brier score}
\end{Highlighting}
\end{Shaded}

\begin{verbatim}
## [1] 0.1738697
\end{verbatim}

\begin{Shaded}
\begin{Highlighting}[]
\NormalTok{tab\_test }\OtherTok{\textless{}{-}} \FunctionTok{prop.table}\NormalTok{(}\FunctionTok{table}\NormalTok{(MASQ[}\SpecialCharTok{{-}}\NormalTok{train, }\StringTok{"D\_DEPDYS"}\NormalTok{], ct\_preds\_test }\SpecialCharTok{\textgreater{}}\NormalTok{ .}\DecValTok{5}\NormalTok{)) }\DocumentationTok{\#\# confusion matrix}
\DecValTok{1} \SpecialCharTok{{-}} \FunctionTok{sum}\NormalTok{(}\FunctionTok{diag}\NormalTok{(tab\_test)) }\DocumentationTok{\#\# MCR}
\end{Highlighting}
\end{Shaded}

\begin{verbatim}
## [1] 0.2388889
\end{verbatim}

The conditional inference tree provided best predictive accuracy of the
single trees.

\begin{Shaded}
\begin{Highlighting}[]
\DocumentationTok{\#\# AUC on test observations}
\FunctionTok{plot}\NormalTok{(}\FunctionTok{roc}\NormalTok{(}\AttributeTok{resp =}\NormalTok{ y\_test, }\AttributeTok{pred =}\NormalTok{ ct\_preds\_test))}
\end{Highlighting}
\end{Shaded}

\begin{verbatim}
## Setting levels: control = 0, case = 1
\end{verbatim}

\begin{verbatim}
## Setting direction: controls < cases
\end{verbatim}

\includegraphics{Answers-to-exercises_files/figure-latex/unnamed-chunk-9-1.pdf}

\begin{Shaded}
\begin{Highlighting}[]
\FunctionTok{auc}\NormalTok{(}\AttributeTok{resp =}\NormalTok{ y\_test, }\AttributeTok{pred =}\NormalTok{ ct\_preds\_test)}
\end{Highlighting}
\end{Shaded}

\begin{verbatim}
## Setting levels: control = 0, case = 1
## Setting direction: controls < cases
\end{verbatim}

\begin{verbatim}
## Area under the curve: 0.805
\end{verbatim}

\end{document}
